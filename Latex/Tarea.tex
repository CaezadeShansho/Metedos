\documentclass{udparticle}
\headertext{Metodos Numéricos}
\title{Metodos Numéricos : Tarea 1}
\author{Thomas Muñoz , Diego Vilches , Javiera Araya , Ignacio Yanjari.}
\usepackage{graphicx}
\usepackage{float}
\usepackage{array}
\graphicspath{ {images/} }
\begin{document}
\maketitle
\begin{enumerate}

\item Usando los métodos de bisección, falsa posición, y secante, encuentre la raíz aproximada 
de las siguientes ecuaciones no lineales en los intervalos indicados:

\begin{enumerate}
    

\item  \(x^3 - 3sen(x) +1 = 0\) , sobre [0,2].
    \begin{table}[H]
    \centering
        \begin{tabular} { |a|a|a|a|}
        
        \hline
        Métodos       & Secante & Biseccion & Falsa Posicion  \\
        \hline
        Cero Obtenido &  -1,5873       &    0,3558       &      -1,5873           \\
        \hline
        Iteraciones   &     6        &      15     &        7         \\
        \hline
        
        \end{tabular}
    \end{table}
    
\item \( e^{-t/2} cos(4t) = 0 \), sobre [0,1].
    \begin{table}[H]
    \centering
        \begin{tabular} { |a|a|a|a|}
        
        \hline
        Métodos       & Secante & Biseccion & Falsa Posicion  \\
        \hline
        Cero Obtenido &  1,9635       &    0,3927       &      0,3927           \\
        \hline
        Iteraciones   &     3        &      12     &        4         \\
        \hline
        
        \end{tabular}
    \end{table}

\item \(x + 40 -x\cosh(\frac{60}{x}) = 0 \), sobre [40,60].
    \begin{table}[H]
    \centering
        \begin{tabular} { |a|a|a|a|}
        
        \hline
        Métodos       & Secante & Biseccion & Falsa Posicion  \\
        \hline
        Cero Obtenido &  50,5399       &    50,5399       &      50,5399           \\
        \hline
        Iteraciones   &     5        &      15     &        9         \\
        \hline
        
        \end{tabular}
    \end{table}

\item \(e^{0.5x}\cos(0.05\sqrt{200-\frac{x^2}{10}}) -1 = 0 \), sobre [0,4].
    \begin{table}[H]
    \centering
        \begin{tabular} { |a|a|a|a|}
        
        \hline
        Métodos       & Secante & Biseccion & Falsa Posicion  \\
        \hline
        Cero Obtenido &  50,5481       &    50,5481       &      50,5481           \\
        \hline
        Iteraciones   &     5        &      13     &        18         \\
        \hline
        
        \end{tabular}
    \end{table}

\item \(f (\theta) = \frac{0.6\sen{\theta}}{\sqrt{(cos(\theta) - 0.6)^2 + sen(\theta)^2}} -  \frac{0.6\sen{\theta}}{\sqrt{(cos(\theta) + 0.6)^2 + sen(\theta)^2}} = 0, \theta \in [1,2].\)
(sol. exacta \theta^* = \frac{\pi}{2}) 

    \begin{table}[H]
    \centering
        \begin{tabular} { |a|a|a|a|}
        
        \hline
        Métodos       & Secante & Biseccion & Falsa Posicion  \\
        \hline
        Cero Obtenido &  1,5708       &   1,5708       &      1,5708           \\
        \hline
        Iteraciones   &     3        &      10     &        2         \\
        \hline
        
        \end{tabular}
    \end{table}
    
\\Con una tolerancia de \(10^{-5}\). Haga una comparación de los métodos en cuanto a la cantidad de iteraciones, el error cometido. Cuál de ellos fue más eficiente?

\end{enumerate}

\item %pregunta 2

\item Considere la ecuación no lineal $f(x) = -x^{3} - \cos(x) = 0$
    \begin{enumerate}
    
        \item Usando el método de Newton encontrar la raı́z próxima al valor $x_{0} = -1$, con una precisión de $10^{-5}$.\\
        \begin{table}[H]
        \centering
        \begin{tabular} { |a|a|}
        
        \hline
        Cero Obtenido &  -0.8655\\
        \hline
        Iteraciones   &    4\\
        \hline
        
        \end{tabular}
        \end{table}
        
        \item Repetir el proceso con el método de Newton modificado, esto es, con la iteración $$x_{n+1} = x_{n} - \frac {f(x_{n})} {f'(x_{0})} $$
        \begin{table}[H]
        \centering
        \begin{tabular} { |a|a|}
        
        \hline
        Cero Obtenido &  -0.8655\\
        \hline
        Iteraciones   &    7\\
        \hline
        
        \end{tabular}
        \end{table}
        ¿Qué método converge más rápido?
        El método de Newton usual converge más rápido, ya que solo tomó 4 iteraciones %debo graficar y argumentar.
    \end{enumerate}

\end{enumerate}
\end{document}
