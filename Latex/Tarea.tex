\documentclass{udparticle}
\headertext{Metodos Numéricos}
\title{Metodos Numéricos : Tarea 1}
\author{Thomas Muñoz , Diego Vilches , Javiera Araya , Ignacio Yanjari.}
\usepackage{graphicx}
\usepackage{float}
\graphicspath{ {images/} }
\begin{document}
\maketitle
\begin{enumerate}

\item Usando los métodos de bisección, falsa posición, y secante, encuentre la raíz aproximada 
de las siguientes ecuaciones no lineales en los intervalos indicados:

\begin{enumerate}
    

\item  \(x^3 - 3sen(x) +1 = 0\) , sobre [0,2].

\item \( e^{-t/2} cos(4t) = 0 \), sobre [0,1].

\item \(x + 40 -x\cosh(\frac{60}{x}) = 0 \), sobre [40,60].

\item \(e^{0.5x}\cos(0.05\sqrt{200-\frac{x^2}{10}}) -1 = 0 \), sobre [0,4].

\item \(f (\theta) = \frac{0.6\sen{\theta}}{\sqrt{(cos(\theta) - 0.6)^2 + sen(\theta)^2}} -  \frac{0.6\sen{\theta}}{\sqrt{(cos(\theta) + 0.6)^2 + sen(\theta)^2}} = 0, \theta \in [1,2].\)
(sol. exacta \theta^* = \frac{\pi}{2}) \\
\(\)con  una   tolerancia de \(10^-5\). Haga una
comparación de los métodos en cuanto a la cantidad
de iteraciones, el error cometido. Cuál de ellos 
fue más eficiente?

\end{enumerate}




\end{enumerate}
\end{document}
