\documentclass{udparticle}
\headertext{Metodos Numéricos}
\title{Metodos Numéricos : Tarea 1}
\author{Thomas Muñoz , Diego Vilches , Javiera Araya , Ignacio Yanjari.}
\usepackage{graphicx}
\usepackage{float}
\usepackage{array}
\graphicspath{ {images/} }

\begin{document}
\maketitle

\begin{enumerate}

\item Usando los métodos de bisección, falsa posición, y secante, encuentre la raíz aproximada 
de las siguientes ecuaciones no lineales en los intervalos indicados:

\begin{enumerate}
    

\item  \(x^3 - 3sen(x) +1 = 0\) , sobre [0,2].
    \begin{table}[H]
    \centering
        \begin{tabular} { |c|c|c|c|}
        
        \hline
        Métodos       & Secante & Biseccion & Falsa Posicion  \\
        \hline
        Cero Obtenido &  -1,5873       &    0,3558       &      -1,5873           \\
        \hline
        Iteraciones   &     6        &      15     &        7         \\
        \hline
        
        \end{tabular}
    \end{table}
    
\item \( e^{-t/2} cos(4t) = 0 \), sobre [0,1].
    \begin{table}[H]
    \centering
        \begin{tabular} { |c|c|c|c|}
        
        \hline
        Métodos       & Secante & Biseccion & Falsa Posicion  \\
        \hline
        Cero Obtenido &  1,9635       &    0,3927       &      0,3927           \\
        \hline
        Iteraciones   &     3        &      12     &        4         \\
        \hline
        
        \end{tabular}
    \end{table}

\item \(x + 40 -x\cosh(\frac{60}{x}) = 0 \), sobre [40,60].
    \begin{table}[H]
    \centering
        \begin{tabular} { |c|c|c|c|}
        
        \hline
        Métodos       & Secante & Biseccion & Falsa Posicion  \\
        \hline
        Cero Obtenido &  50,5399       &    50,5399       &      50,5399           \\
        \hline
        Iteraciones   &     5        &      15     &        9         \\
        \hline
        
        \end{tabular}
    \end{table}

\item \(e^{0.5x}\cos(0.05\sqrt{200-\frac{x^2}{10}}) -1 = 0 \), sobre [0,4].
    \begin{table}[H]
    \centering
        \begin{tabular} { |c|c|c|c|}
        
        \hline
        Métodos       & Secante & Biseccion & Falsa Posicion  \\
        \hline
        Cero Obtenido &  50,5481       &    50,5481       &      50,5481           \\
        \hline
        Iteraciones   &     5        &      13     &        18         \\
        \hline
        
        \end{tabular}
    \end{table}

\item $f (\theta) = \frac{0.6\sen{\theta}}{\sqrt{(cos(\theta) - 0.6)^2 + sen(\theta)^2}} -  \frac{0.6\sen{\theta}}{\sqrt{(cos(\theta) + 0.6)^2 + sen(\theta)^2}} = 0, \theta \in [1,2].
(sol. exacta \theta^* = \frac{\pi}{2})$ 

    \begin{table}[H]
    \centering
        \begin{tabular} { |c|c|c|c|}
        
        \hline
        Métodos       & Secante & Biseccion & Falsa Posicion  \\
        \hline
        Cero Obtenido &  1,5708       &   1,5708       &      1,5708           \\
        \hline
        Iteraciones   &     3        &      10     &        2         \\
        \hline
        
        \end{tabular}
    \end{table}
    
Con una tolerancia de $10^{-5}$. Haga una comparación de los métodos en cuanto a la cantidad de iteraciones, el error cometido. Cuál de ellos fue más eficiente?

\end{enumerate}

%pregunta 2
%pendiente: poner gráfica
\item Considere la ecuación no lineal $f(x)= \frac{1}{2}+\frac{1}{4}x^2-xsen(x)-\frac{1}{2}cos(2x)=0$
	\begin{enumerate}
	\item  Usando el método de Newton con punto inicial $x_{0}=\frac{\pi}{2}$ encuentre la solución aproximada de $f$. Para ello REalize las iteraciones necesarias hasta que se cumpla el criterio de parada %$\abs{x_{n+1}-x_{n}} \l 10^{-6} $
%pendiente: valor absoluto

		\begin{table} [H]
			\centering
			\begin{tabular}{|c|c|c|}
				\hline
				$x_{0}$ & Cero Obtenido & Iteraciones\\
				\hline
				$\frac{\pi}{2} $ & 1,8955 & 18\\
				\hline 
			\end{tabular}
		\end{table}
	
	\item Repita el proceso tomando como valores iniciales $x_{0}=5\pi$ y $x_{0}=10\pi$ ¿La sucesión construída con estos puntos iniciales converge?
	 	\begin{table} [H]
			\centering
			\begin{tabular}{|c|c|c|}
				\hline
				$x_{0}$ & Cero Obtenido & Iteraciones\\
				\hline
				$5\pi$ & 1,8955 & 22 \\
				\hline 
				$10\pi$ & 0 & 66\\
				\hline
			\end{tabular}
		\end{table}
		
La sucesión, en el caso de $x_{0}=5\pi$, converge al mismo cero que en el ejercicio anterior. Por otra parte, cuando $x_{0}=10\pi$, esta converge a 0.

	\item Use el método de la secante para encontrar la solución aproximada tomando como puntos iniciales $x_{0}=\frac{\pi}{2}$ y $x_{0}=5\pi$, como criterio de parada el mismo descrito en (a).
	\begin{table} [H]
			\centering
			\begin{tabular}{|c|c|c|c|}
				\hline
				$x_{0}$& $x_{1}$ & Cero Obtenido & Iteraciones\\
				\hline
				$\frac{\pi}{2}$ & 1,7854 & 1,8955 & 24 \\
				\hline 
				$5\pi$& 13,0900 & 0 & 31\\
				\hline
			\end{tabular}
		\end{table}
		
Se puede apreciar que si $x_{0}=5\pi$, al ocupar el método de la secantem se obtiene un cero distinto a cuando se ocupa el método de Newton.	
	\end{enumerate}
		

\item Considere la ecuación no lineal $f(x) = -x^{3} - \cos(x) = 0$
    \begin{enumerate}
    
        \item Usando el método de Newton encontrar la raíz próxima al valor $x_{0}=-1$, con una precisión de $10^{-5}$.\\
        \begin{table}[H]
        \centering
        \begin{tabular} { |c|c|}
        
        \hline
        Cero Obtenido &  -0.8655\\
        \hline
        Iteraciones   &    4\\
        \hline
        
        \end{tabular}
        \end{table}
        
        \item Repetir el proceso con el método de Newton modificado, esto es, con la iteración $$x_{n+1} = x_{n} - \frac {f(x_{n})} {f'(x_{0})} $$
        \begin{table}[H]
        \centering
        \begin{tabular} { |c|c|}
        
        \hline
        Cero Obtenido &  -0.8655\\
        \hline
        Iteraciones   &    7\\
        \hline
        
        \end{tabular}
        \end{table}
        ¿Qué método converge más rápido?
        El método de Newton usual converge más rápido, ya que solo tomó 4 iteraciones %debo graficar y argumentar.
    \end{enumerate}

\item El dinero necesario para pagar la cuota correspondiente a un crédito hipotecario a interés fijo se suele
estimar mediante la denominada “ecuación de la anualidad ordinaria”:
\begin{center}
    $ Q = \frac{A}{i}(1-(1+i)^{-n}) $
\end{center}
donde Q es la cantidad pedida en préstamo, A es la cuota que debe pagar el beneficiario por el
préstamo, i es la tasa de interés fijado por la entidad bancaria que concede el préstamo y n es el
número de periodos durante los cuales se realizan pagos de la cuota.
Una pareja que desea comenzar una vida en común se plantea adquirir una vivienda y para ello saben
que necesitan pedir un préstamo de 15000 dólares a pagar semestralmente durante un plazo de 10 años.
Sabiendo que para atender este pago pueden destinar una cantidad máxima de 200 dólares mensuales,
calcule cual es el tipo máximo de interés al que pueden negociar su préstamo con las entidades bancarias.
Hint.- Usar método de Newton, tomando como punto inicial i0 = 0.03.
Suponga ahora que desean endeudarse en 15 años en lugar de 10. Cual sería el interés en esta situación?


\item Considere la función \(f(x) = x*cos(x)-e^x+ 1\). % pregunta 5 
\begin{enumerate}
    
\vspace{0.9cm}
\item Considere las siguientes funciones. Realice unas 12 iteraciones de punto fijo, usando como puntos iniciales x0 = 0.5 y x0 = 0.5.\\ 


\begin{equation}
 g1(x) = \frac{e^x+x-1}
{1 + cos(x)}
\end{equation}
\\
\begin{itemize}
\item x0=0.5
\end{itemize}


\begin{table}[H]
    \centering
        \begin{tabular} { |c|c|}
        
        \hline
        iteración  &  Punto\\
        \hline
        1 &  0,6118        \\
         \hline
        2 &  0,8004        \\
         \hline
        3 &  1,1947        \\
         \hline
        4 &  2,5579        \\
         \hline
        5 &  87,3616        \\
         \hline
        6 & 4,7832e+37        \\
         \hline
        7 & inf           \\
         \hline
        8 &  NaN        \\
         \hline
        9 &  NaN        \\
         \hline
        10 &  NaN        \\
         \hline
        11 &  NaN        \\
         \hline
        12 &  NaN        \\
        \hline
        
        \end{tabular}
    \end{table}
 
 \begin{itemize}
\item x0=-0.5
\end{itemize}
\begin{table}[H]
    \centering
        \begin{tabular} { |c|c|}
        
        \hline
        iteración  &  Punto\\
        \hline
        1 &  -0.4759        \\
         \hline
        2 &  -0.4524        \\
         \hline
        3 &  -0.4298        \\
         \hline
        4 &   -0.4081      \\
         \hline
        5 &  -0.3875        \\
         \hline
        6 &  -0.3680       \\
         \hline
        7 &  -0.3497        \\
         \hline
        8 &  -0.3324        \\
         \hline
        9 &  -0.3163        \\
         \hline
        10 &  -0.3012       \\
         \hline
        11 &  -0.2871       \\
         \hline
        12 &  -0.2739        \\
        \hline
        \end{tabular}
\end{table}

\begin{equation}
 g2(x) = \frac{\sqrt{x(e^x-1)}}
{cos(x)}
\end{equation}
\\
\begin{itemize}
\item x0=0.5
\end{itemize}

\begin{table}[H]
    \centering
        \begin{tabular} { |c|c|}
        
        \hline
        iteración  &  Punto\\
        \hline
        1 &  0.6080       \\
         \hline
        2 &   0.7872     \\
         \hline
        3 &  1.1555       \\
         \hline
        4 &   2.4964     \\
         \hline
        5 & 0.0000 + 5.8994i        \\
         \hline
        6 &  0.1106 - 0.0106i       \\
         \hline
        7 & 0.1140 - 0.0113i         \\
         \hline
        8 &    0.1177 - 0.0121i     \\
         \hline
        9 &     0.1216 - 0.0130i     \\
         \hline
        10 &   0.1258 - 0.0140i       \\
         \hline
        11 &    0.1303 - 0.0151i   \\
         \hline
        12 &  0.1351 - 0.0163i        \\
        \hline
        
        \end{tabular}
        
    \end{table}
    \begin{itemize}
\item x0=-0.5
\end{itemize}

\begin{table}[H]
    \centering
        \begin{tabular} { |c|c|}
        
        \hline
        iteración  &  Punto\\
        \hline
        1 &  0.4735       \\
         \hline
        2 &   0.5676     \\
         \hline
        3 &  0.7171
       \\
         \hline
        4 &   0.9989     \\
         \hline
        5 & 1.7791        \\
         \hline
        6 &  0.0000 + 6.5089i      \\
         \hline
        7 &   0.0037 - 0.0660i       \\
         \hline
        8 &     0.0026 - 0.0660i   \\
         \hline
        9 &   0.0015 - 0.0660i       \\
         \hline
        10 &    0.0004 - 0.0660i      \\
         \hline
        11 &    0.0007 + 0.0659i \\
         \hline
        12 &   0.0004 - 0.0659i       \\
        \hline
        
        \end{tabular}
        
    \end{table}
 \vspace{3cm}   
 \item Teniendo en cuenta las siguientes funciones de iteración de punto fijo.  Realize unas 12 iteraciones de punto fijo, usando como puntos iniciales x0 = 0.5 y x0 = 0.5.\\   
 
 \begin{equation}
 g3(x) = x-\frac{f(x)}
{f'(x)}
\end{equation}
\\
 \begin{itemize}
\item x0=0.5
\end{itemize}

\begin{table}[H]
    \centering
        \begin{tabular} { |c|c|}
        
        \hline
        iteración  &  Punto\\
        \hline
        1 &   0.2923       \\
         \hline
        2 &     0.1645   \\
         \hline
        3 &  0.0891 \\
         \hline
        4 &  0.0468     \\
         \hline
        5 &    0.0241    \\
         \hline
        6 & 0.0122       \\
         \hline
        7 &    0.0062 \\
         \hline
        8 &  0.0031     \\
         \hline
        9 &      0.0015    \\
         \hline
        10 &    7.7566e-04     \\
         \hline
        11 &   3.8803e-04  \\
         \hline
        12 &   1.9407e-04      \\
        \hline
        
        \end{tabular}
        
    \end{table}
 \begin{itemize}
\item x0=-0.5
\end{itemize}

\begin{table}[H]
    \centering
        \begin{tabular} { |c|c|}
        
        \hline
        iteración  &  Punto\\
        \hline
        1 &  0.9462        \\
         \hline
        2 &     0.5755  \\
         \hline
        3 &    0.3398        \\
 
         \hline
        4 &   0.1932  \\
         \hline
        5 &  0.1057       \\
         \hline
        6 &      0.0560  \\
         \hline
        7 & 0.0289    \\
         \hline
        8 & 0.0147      \\
         \hline
        9 &   0.0074       \\
         \hline
        10 &    0.0037     \\
         \hline
        11 &   0.0019  \\
         \hline
        12 &   9.3724e-04     \\
        \hline
        
        \end{tabular}
        
    \end{table}
 \vspace{2cm}
     \begin{equation}
 g4(x) = x-2\frac{f(x)}
{f'(x)}
\end{equation}
\\
 \begin{itemize}
\item x0=0.5
\end{itemize}

\begin{table}[H]
    \centering
        \begin{tabular} { |c|c|}
        
        \hline
        iteración  &  Punto\\
        \hline
        1 &      0.0846   \\
         \hline
        2 &    0.0041    \\
         \hline
        3 &   1.1230e-05 \\
         \hline
        4 &   8.4146e-11    \\
         \hline
        5 &   8.4146e-11   \\
         \hline
        6 &   8.4146e-11     \\
         \hline
        7 & 8.4146e-11   \\
         \hline
        8 &  8.4146e-11   \\
         \hline
        9 &   8.4146e-11      \\
         \hline
        10 &   8.4146e-11      \\
         \hline
        11 & 8.4146e-11   \\
         \hline
        12 &  8.4146e-11      \\
        \hline
        
        \end{tabular}
        
    \end{table}
     \begin{itemize}
\item x0=-0.5
\end{itemize}

\begin{table}[H]
    \centering
        \begin{tabular} { |c|c|}
        
        \hline
        iteración  &  Punto\\
        \hline
        1 &     2.3924  \\
         \hline
        2 &    0.6344   \\
         \hline
        3 &   0.1196 \\
         \hline
        4 &    0.0078   \\
         \hline
        5 &   3.9737e-05  \\
         \hline
        6 &   1.0509e-09    \\
         \hline
        7 & 1.0509e-09   \\
         \hline
        8 &  1.0509e-09  \\
         \hline
        9 &  1.0509e-09      \\
         \hline
        10 &  1.0509e-09      \\
         \hline
        11 & 1.0509e-09   \\
         \hline
        12 &  1.0509e-09     \\
        \hline
        
        \end{tabular}
        
    \end{table}
    \vspace{3cm}
    \begin{equation}
 g5(x) = x-\frac{f(x)f'(x)}
{(f'(x))^2-f(x)f''(x)}
\end{equation}
\\
 \begin{itemize}
\item x0=0.5
\end{itemize}

\begin{table}[H]
    \centering
        \begin{tabular} { |c|c|}
        
        \hline
        iteración  &  Punto\\
        \hline
        1 &  -0.0551      \\
         \hline
        2 &    -0.0023   \\
         \hline
        3 &  -3.5887e-06 \\
         \hline
        4 &  1.2916e-10    \\
         \hline
        5 &    1.2916e-10  \\
         \hline
        6 &  1.2916e-10     \\
         \hline
        7 &     1.2916e-10 \\
         \hline
        8 &  1.2916e-10    \\
         \hline
        9 &       1.2916e-10   \\
         \hline
        10 &    1.2916e-10     \\
         \hline
        11 &    1.2916e-10 \\
         \hline
        12 &   1.2916e-10  \\
        \hline
        
        \end{tabular}
        
    \end{table}
     \begin{itemize}
\item x0=0.5
\end{itemize}

\begin{table}[H]
    \centering
        \begin{tabular} { |c|c|}
        
        \hline
        iteración  &  Punto\\
        \hline
        1 &  -0.4614      \\
         \hline
        2 &   -0.3825   \\
         \hline
        3 &  -0.2366 \\
         \hline
        4 &  -0.0667    \\
         \hline
        5 &   -0.0035 \\
         \hline
        6 & -8.1704e-06     \\
         \hline
        7 &     -2.6231e-11 \\
         \hline
        8 &   -2.6231e-11   \\
         \hline
        9 &        -2.6231e-11 \\
         \hline
        10 &     -2.6231e-11    \\
         \hline
        11 &    -2.6231e-11 \\
         \hline
        12 &    -2.6231e-11  \\
        \hline
        
        \end{tabular}
        
    \end{table}
\vspace{3cm}
\item Que puede decir sobre el comportamiento de las iteraciones de punto fijo calculadas anteriormente.
\begin{itemize}

\item De la función g1(x), cuando se toma el punto inicial x0=0.5, la iteración por punto fijo diverge, pues, cada vez que se hacen más iteraciones, el punto fijo tendera a infinito. 
\end{itemize}
\end{enumerate}
<<<<<<< HEAD

=======
\end{enumerate}
>>>>>>> 3d7d430da91f5b054d997324bbd5b2ed5236e973
\end{document}
